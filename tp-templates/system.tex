\chapter{System Test Plan}
\section{TSS-01 for Movement Mechanics }
\begin{itemize}
\item Goal: Determine that the jumping mechanics for the character in Line
Bounce work robustly and accurately. 
\item Resources Required: User (tester), iPhone, Android Phone, Computer,
Time. 
\item Structure of the Test Plan (Applicable TES list): 

\begin{itemize}
\item FUN-01 (game start movement) 
\item pausing (FUN-18) 
\item bouncing behavior 

\begin{itemize}
\item FUN-02, FUN-45 (top of lines, velocity change) 
\item FUN-05, FUN-45 (bottom of lines, velocity change) 
\item FUN-06, FUN-19 (bouncing off walls, velocity change) 
\item FUN-23 (platform size affecting velocity) 
\end{itemize}
\item off-screen movement 

\begin{itemize}
\item FUN-03 (falling off (bottom of) screen) 
\item FUN-04 (moving up (top of) screen) 
\end{itemize}
\item collisions 

\begin{itemize}
\item FUN-11, FUN-22 (power-up collection, activation, -doesn\textquoteright{}t
affect movement-) 
\item FUN-03 (touching enemies falling off screen ending game) 
\item FUN-46 (movement affecting score) 
\end{itemize}
\end{itemize}
\item Resources needed: 

\begin{itemize}
\item Relevant devices (iPhone/iPad, Android, facebook accessable PC and
web browser) 
\item Timing software 
\item Trajectory measuring software
\end{itemize}
\end{itemize}

\subsection{TES-1.1: Game Start State }
\begin{itemize}
\item Goal: Show that the avatar, at the start of a game, will start at
the top center of the user\textquoteright{}s screen and accelerate
in the -y direction only 

\begin{itemize}
\item Set a time bound (?) 
\end{itemize}
\item Testcases: 

\begin{itemize}
\item Try starting game with several different avatar/power up combinations 
\item Try pausing game, ending it, and then starting it again (does the
game state get cleared when starting new game?) 
\end{itemize}
\item Capture the output 

\begin{itemize}
\item Measure elapsed time 
\item Measure start location 
\item Measure acceleration 
\end{itemize}
\item Evaluate the output data 

\begin{itemize}
\item compare to expected data (in terms of the goals) 
\end{itemize}
\item Generate report 

\begin{itemize}
\item Produce summary of generated reports 
\item Evaluate summary with respect to goal
\end{itemize}
\end{itemize}

\subsection{TES-1.2: Pausing }
\begin{itemize}
\item Goal: Show that, while in a pause state, all moving objects halt in
place, but exiting the pause state will restore original movement 

\begin{itemize}
\item resumes gameplay in a coherent fashion that will not break the user\textquoteright{}s
immersion in the game 
\item objects originally moving in a certain trajectory will not suddenly
change trajectory for no reason (see TES-1.3) 
\item Game state will be saved (see TES 2.1) 
\end{itemize}
\item Testcases: 

\begin{itemize}
\item try tapping the pause/resume button on several different occasions
coinciding with various events 
\item try tapping the button at various time interval increments (ex. 50
ms, 1 s, 2 s, \ldots{} ) 
\end{itemize}
\item Capture the output 

\begin{itemize}
\item measure trajectories before entering/after leaving pause state 
\item measure elapsed response time and record qualitative response behavior 
\end{itemize}
\item Evaluate the numeric result 

\begin{itemize}
\item compare trajectories 
\item evaluate response time and behavior with those expected 
\end{itemize}
\item Generate report 

\begin{itemize}
\item Produce summary of generated reports with regard to data collection 
\item Evaluate summary with respect to goal
\end{itemize}
\end{itemize}

\subsection{TES-1.3: Bouncing Behavior }
\label{bounce}
\begin{itemize}
\item Goal: Show that the avatar will bounce off of objects in game, that
its trajectory changes in a fun and intuitive fashion 

\begin{itemize}
\item avatar will only change from -y to +y velocity when colliding with
the top of drawn lines (see FUN-02, FUN-45) 
\item no component of avatar\textquoteright{}s velocity will be affected
upon colliding with the bottoms of lines (see FUN-05, FUN-45) 
\item side walls (left and right) are \textquotedblleft{}solid\textquotedblright{}:
avatar will reverse x velocity upon collision (see FUN-06, FUN-19) 
\item shorter drawn platforms will give more of a boost than longer drawn
ones (see FUN-23) 
\end{itemize}
\item Testcases: 

\begin{itemize}
\item test various different environments in isolated conditions from the
bullets described in goals (ex. avatar with a velocity such that it
will collide with the top of a line only) 
\item test various different environments in various conditions from the
bullets described in goals (ex. avatar with a velocity such that it
will collide with the bottom of a line after colliding with the top
of a line) 
\item test the above two with different combinations of avatars, power-ups
that are in active/inactive statuses, arbitrarily placed lines (ranging
from extremely condensed to sparsely place ones) 
\end{itemize}
\item Capture the output 

\begin{itemize}
\item measure trajectories before/after each collision 
\item measure elapsed response time and record qualitative response behavior 
\end{itemize}
\item Evaluate the numeric result 

\begin{itemize}
\item organize data and compare trajectories with relevant goals 
\item evaluate response time and behavior with relevant goals 
\end{itemize}
\item Generate report 

\begin{itemize}
\item Produce summary of generated reports with regard to data collection 
\item Evaluate summary with respect to goal
\end{itemize}
\end{itemize}

\subsection{TES-1.4: Off-Screen Movement}

\subsection{TES-1.5: Collisions}

\subsection{TES-1.6: Movement Affecting Score}

\section{TSS-02 for Game State Mechanics }
\begin{itemize}
\item Goal: Determine that the mechanics that govern when the game is started
and lost work robustly and accurately.
\item Resources Required: 

\begin{itemize}
\item User Hours 
\item iPhone 
\item Android Phone 
\item Computer 
\end{itemize}
\item Structure of Test Plan: 

\begin{itemize}
\item Pause Button Pauses Game: FUN-18 

\begin{itemize}
\item Player can end, resume or restart game from Pause menu 
\end{itemize}
\item User loses when touching lethal hazard 

\begin{itemize}
\item Contact with enemies: ACC-07 
\item Falling through bottom: FUN-03 
\end{itemize}
\item Results of changing application state 

\begin{itemize}
\item Backgrounding the app mid-playthrough pauses game: FUN-40 
\item Exiting the app mid-playthrough loses the game: FUN-41 
\end{itemize}
\item In an optimal game the player climbs forever: ACC-05 

\begin{itemize}
\item Game should not transition to an end-state by climbing too high
\end{itemize}
\end{itemize}
\end{itemize}

\subsection{TES-2.1: Pause Menu }
\begin{itemize}
\item Goal: Show that the the Player can pause the game at any time during
a playthrough, which freezes game state. Show that the user can terminate
the game or restart their playthrough from the pause menu.
\item Test cases: 

\begin{itemize}
\item User manually starts game, presses pause button. 
\item From pause menu, user attempts to quit game. 
\item From pause menu, user attempts to restart game.
\item All of this will be performed manually by a tester, without using additional software.
\end{itemize}
\item Capture the output 

\begin{itemize}
\item Collect data on how game state responded to operations 
\end{itemize}
\item Evaluate the result: 

\begin{itemize}
\item Pressing pause should bring up pause menu. 
\item Attempting to quit game should exit from game, to game over screen. 
\item Attempting to restart game should begin a new playthrough. 
\end{itemize}
\item Generate report 

\begin{itemize}
\item Produce summary of generated reports with regard to data collection 
\item Evaluate summary with respect to goal
\end{itemize}
\end{itemize}

\subsection{TES-2.2: Lethal Hazards }
\label{hazard}
\begin{itemize}
\item Goal: Show that contact of the player\textquoteright{}s avatar with
lethal hazards results in a lost game state as intended. 
\item Test cases: 

\begin{itemize}
\item User\textquoteright{}s avatar makes contact with enemy avatar. 
\item User\textquoteright{}s avatar makes contact with bottom of screen. 
\item All of this will be performed manually by a tester without the use of additional software.
\end{itemize}
\item Capture the output 

\begin{itemize}
\item Collect data on how game state responded to operations 
\end{itemize}
\item Evaluate the result: 

\begin{itemize}
\item Both cases should result in a lost game state. 
\end{itemize}
\item Generate report 

\begin{itemize}
\item Produce summary of generated reports with regard to data collection 
\item Evaluate summary with respect to goal
\end{itemize}
\end{itemize}

\subsection{TES-2.3: Change Application State }
\begin{itemize}
\item Goal: Show that changing the state of the application mid-playthrough
produces the intended result relative to the state of the current
playthrough. 
\item Test cases: 

\begin{itemize}
\item Background the Line Bounce application mid-playthrough. 
\item Exit the Line Bounce application mid-playthrough. 
\item All of this will be performed manually by a tester without the use of additional software.
\end{itemize}
\item Capture the output 

\begin{itemize}
\item Collect data on how game state responded to operations 
\end{itemize}
\item Evaluate the result: 

\begin{itemize}
\item Backgrounding the app mid-playthrough should pause the game, and bring
up the Pause Menu. 
\item Exiting the app mid-playthrough should exit the game. 
\end{itemize}
\item Generate Report 

\begin{itemize}
\item Produce summary of generated reports with regard to data collection 
\item Evaluate summary with respect to goal
\end{itemize}
\end{itemize}

\subsection{TES-2.4: Climbing High }
\begin{itemize}
\item Goal: Show that climbing to a high altitude by itself does not change
the game state. 
\item Test cases: 

\begin{itemize}
\item In a playthrough, climb to 10{[}a{]},000 feet. 
\item In a playthrough, climb to 100,000 feet. 
\item All of this will be performed manually by a tester without the use of additional software.
\end{itemize}
\item Capture the output 

\begin{itemize}
\item Collect data on how game state responded to operations 
\end{itemize}
\item Evaluate the result: 

\begin{itemize}
\item In both cases, no matter how high the avatar climbs, game should not
end. 
\end{itemize}
\item Generate Report 

\begin{itemize}
\item Produce summary of generated reports with regard to data collection 
\item Evaluate summary with respect to goal
\end{itemize}
\end{itemize}

\section{TSS-03 for Menus{[}b{]} }
\begin{itemize}
\item Goal: Determine that the menus function according to the requirements.
(Note: The Store menus will be covered in the Store TSS.)
\item Resources Required: 

\begin{itemize}
\item User Hours 
\item iPhone 
\item Android Phone 
\item Computer 
\item Internet Connection 
\item Server
\end{itemize}
\item Structure of the Test Plan: 

\begin{itemize}
\item Menu features are present and work correctly 

\begin{itemize}
\item The Main Menu should feature many options: FUN-09 
\item The Settings Menu should allow users to change options: FUN-17 
\item The start screen will give the option of connecting to Facebook before
going to the Main Menu: FUN-33 
\item The user can logout of Facebook from the Settings Menu: FUN-36 
\item Pressing the pause button will bring up a menu that allows the user
to perform a few actions: FUN-39 
\item Buttons in menus should always perform desired functionality: ROB-08 
\end{itemize}
\item The menu will have fast response time: PER-04
\end{itemize}
\end{itemize}

\subsection{TES-3.1: Menu features are present and function correctly }
\begin{itemize}
\item Goal: Show that the menu features are present in the required locations
and perform the required functions across all platforms. 
\item Test cases: 

\begin{itemize}
\item Check for presence of all menu features 

\begin{itemize}
\item User navigates the game and menus 
\end{itemize}
\item Check functionality of all menu features 

\begin{itemize}
\item User tries to use menu features manually 
\item User writes and runs test code to check back-end functionality 
\end{itemize}
\end{itemize}
\item Capture the output 
\item Evaluate the result: 

\begin{itemize}
\item All required menu features should be present in the required locations 
\item All menu features should function as required 
\end{itemize}
\item Generate report
\end{itemize}

\subsection{TES-3.2: The menu will have fast response time }
\label{menu}
\begin{itemize}
\item Goal: Show that the menu feature response times satisfy the performance
requirements stated in PER-04. 
\item Test cases: 

\begin{itemize}
\item Check the response time of each menu feature 

\begin{itemize}
\item User writes and runs test code that times each menu function 
\end{itemize}
\end{itemize}
\item Capture the output 
\item Evaluate the result: 

\begin{itemize}
\item All menu features should have a response time of under 0.25 seconds
(PER-04 as of 10/18/13) 
\end{itemize}
\item Generate report \end{itemize}
\section{Frontend (what the user sees)}
\subsection{Transitions}
\subsection{Pages}
\subsection{Gameview}
\subsubsection{Scrolling (In-Game)}
\section{Backend - Admin}
\section{Backend - Game}


\section{TSS-09 for Main Menu}
\begin{itemize}
\item Goal: Determine that all aspects of the Main Menu function properly and satisfy
performance requirements.
\item Resources Required: 

\begin{itemize}
\item User Hours 
\item iPhone 
\item Android Phone 
\item Computer 
\item Internet Connection 
\item Server
\end{itemize}
\item Structure of the Test Plan: 

\begin{itemize}

\item Pressing each button sends user to the appropriate screen.
\item All buttons and redirects function in accordance with Performance specifications.

\end{itemize}
\end{itemize}

\subsection{TES-9.1: Menu buttons all redirect the user properly }
\begin{itemize}
\item Goal: Show that all buttons on the menu are functional, and redirect the user to the appropriate page.
\item Test cases: 

\begin{itemize}
\item Test every main menu button to make sure it redirects correctly.

\begin{itemize}
\item Tester manually presses PLAY button.
\item Tester manually presses LEADERBOARDS button.
\item Tester manually presses SHOP button.
\item Tester manually presses INVITE FRIEND button.
\item Tester manually presses Settings icon.

\end{itemize}

\end{itemize}
\item Capture the output
\begin{itemize}
\item Tester manually saves test results into a spreadsheet.
\end{itemize}
\item Evaluate the result: 

\begin{itemize}
\item Pressing PLAY takes user to pre-game powerup selection menu.
\item Pressing LEADERBOARDS takes user to the leaderboards screen.
\item Pressing SHOP takes user to the store.
\item Pressing INVITE FRIEND takes user to the Invite A Friend screen.
\item Pressing Settings icon takes user to Settings.
\end{itemize}
\item Generate report
\end{itemize}

\subsection{TES-9.2: Menu buttons respond with minimal delay}
\begin{itemize}
\item Goal: Show that all buttons on the main menu respond with minimal delay.
\item Test Cases: 
\begin{itemize}
\item \ref{menu}
\end{itemize}
\item Capture the output
\begin{itemize}
\item Tester manually saves test results into a spreadsheet.
\end{itemize}
\item Evaluate the result: 

\begin{itemize}
\item Pressing PLAY takes user to pre-game powerup selection menu.
\item Pressing LEADERBOARDS takes user to the leaderboards screen.
\item Pressing SHOP takes user to the store.
\item Pressing INVITE FRIEND takes user to the Invite A Friend screen.
\item Pressing Settings icon takes user to Settings.
\end{itemize}
\item Generate report
\end{itemize}



\section{TSS-10 for Enemies}
\begin{itemize}
\item Goal: Determine that all aspects regarding Enemies perform as specified in the requirements.
\item Resources Required: 

\begin{itemize}
\item User Hours 
\item iPhone 
\item Android Phone 
\item Computer 
\item Internet Connection 
\item Server
\end{itemize}
\item Structure of the Test Plan: 

\begin{itemize}

\item Enemy generation occurs randomly, but according to a formula: FUN-30 
\item An Enemy is destroyed if the user's avatar makes contact with its top half.
\item Contact with the bottom half of an Enemy causes the player to lose: ACC-07
\item Enemies are stationary.

\end{itemize}
\end{itemize}

\subsection{TES-10.1: Enemy Generation occurs randomly, and follows preset formula correctly }
\begin{itemize}
\item Goal: Show that enemies are generated randomly according to a formula that is based 
on the player's current ingame score and the player's overall experience level.
\item Test cases: 

\begin{itemize}
\item Test enemy generation based on height and level

\begin{itemize}
\item Write code that increments each time an enemy is generated and records the score and user's experience
level at the time of capture.
\item Repeat for a statistically significant number of playthroughs, with a range of user levels.

\end{itemize}

\end{itemize}
\item Capture the output
\begin{itemize}
\item Save all test outputs and collect information into a spreadsheet.
\end{itemize}
\item Evaluate the result: 

\begin{itemize}
\item Compare the formula for enemy generation against the spreadsheet data to determine accuracy.
\end{itemize}
\item Generate report
\end{itemize}

\subsection{TES-10.2: Destroying Enemies }
\begin{itemize}
\item Goal: Show that the functionality of destroying enemies by making contact between
the user's avatar and the top half of the enemy works properly and accurately.
\item Test cases: 

\begin{itemize}
\item Test to make sure enemies are destroyed by touching top half. 

\begin{itemize}
\item User manually plays line bounce and contacts top half of an enemy with the avatar.
\item User repeats this process for every type of enemy that appears ingame. 
\end{itemize}
\end{itemize}
\item Capture the output 
\begin{itemize}
\item User writes down results
\end{itemize}
\item Evaluate the result: 

\begin{itemize}
\item All enemy types should respond to being touched on top with destruction.
\end{itemize}
\item Generate report 
\end{itemize}

\subsection{TES-10.3: Enemies As Hazard }
\begin{itemize}
\item Goal: Show that the functionality of destroying enemies by making contact between
the users avatar and the top half of the enemy works properly and accurately.
\item Test cases: 

\begin{itemize}
\item \ref{hazard}
\item Repeat for each enemy type.
\item This test will be performed manually by a tester without the use of testing software.
\end{itemize}
\item Capture the output 
\begin{itemize}
\item User records results in a spreadsheet.
\end{itemize}
\item Evaluate the result: 

\begin{itemize}
\item If the avatar makes contact with any enemy type from the bottom, it should result in a lost game. 
\end{itemize}
\item Generate report 
\end{itemize}

\subsection{TES-10.4: Enemy Movement }
\begin{itemize}
\item Goal: Show that enemies do not move under any circumstances, and disappear when they are destroyed.
\item Test cases: 

\begin{itemize}
\item Test to make sure enemies don't move.
\begin{itemize}
\item Tester manually plays game and encounters all enemy types.
\item Tester keeps spreadsheet and records enemy behavior before and after destroying that enemy.
\end{itemize}
\end{itemize}
\item Capture the output 
\begin{itemize}
\item User records results in a spreadsheet.
\end{itemize}
\item Evaluate the result: 

\begin{itemize}
\item For each enemy type, that enemy should not move, and should disappear when destroyed.
\end{itemize}
\item Generate report 
\end{itemize}

\section{TSS-11 for Lines}
\begin{itemize}
\item Goal: Determine that all mechanics of lines function properly, and satisfy requirements
for robustness and accuracy.

\item Resources Required:
\begin{itemize}
\item User Hours 
\item iPhone 
\item Android Phone 
\item Computer 
\item Internet Connection 
\item Server
\end{itemize}
\item Structure of the Test Plan: 

\begin{itemize}

\item Line Spool feature functions properly.
\item Bounce mechanics function correctly, including bounce-to-length ratio.
\item Lines do not interact with enemies.

\end{itemize}
\end{itemize}

\subsection{TES-11.1: Line Spool }
\begin{itemize}
\item Goal: Show that Line Spool feature works properly, according to functional specifications.
\item Test cases: 

\begin{itemize}
\item Test that Line Spool only gives the player a length of line equal to the screen width to draw.
\begin{itemize}
\item Write test code to measure screen width and compare it to the length of lines drawn.
\item Tester manually draws longest possible line, which should not exceed one screen width.
\item After using up total Line Spool, tester manually draws more lines. 
\end{itemize}
\end{itemize}
\item Capture the output 
\begin{itemize}
\item User records results in a spreadsheet.
\end{itemize}
\item Evaluate the result: 

\begin{itemize}
\item Total line length drawable at once may not exceed one screen width.
\item When all Line Spool is used up, drawing one additional line should erase the first line drawn
refunding an amount of line spool equal to the length of the erased line. 
\end{itemize}
\item Generate report 
\end{itemize}

\subsection{TES-11.2: Bounce Mechanics }
\begin{itemize}
\item Goal: Show that the Bounce Mechanics regarding lines work properly, including
bounce-to-length ratio. 
\item Test cases: 

\begin{itemize}
\item \ref{bounce}
\end{itemize}
\item Capture the output
\begin{itemize}
\item User records all bounce heights and directions in a spreadsheet.
\end{itemize}
\item Evaluate the result: 
\begin{itemize}
\item \ref{bounce}
\end{itemize}
\item Generate report 
\end{itemize}


\section{Frontend (what the user sees)}
\subsection{Transitions}
\subsection{Pages}
\subsection{Gameview}
\subsubsection{Scrolling (In-Game)}
\section{Backend - Admin}
\section{Backend - Game}
