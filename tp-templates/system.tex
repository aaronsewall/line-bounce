\chapter{System Test Plan}
\section{Frontend (what the user sees)}
\subsection{Transitions}
\subsection{Pages}
\subsection{Gameview}
\subsubsection{Scrolling (In-Game)}
\section{Backend - Admin}
\section{Backend - Game}
\section{TSS-01 for Movement Mechanics }
\begin{itemize}
\item Goal: Determine that the jumping mechanics for the character in Line
Bounce work robustly and accurately. 
\item Resources Required: User (tester), iPhone, Android Phone, Computer,
Time. 
\item Structure of the Test Plan (Applicable TES list): 

\begin{itemize}
\item FUN-01 (game start movement) 
\item pausing (FUN-18) 
\item bouncing behavior 

\begin{itemize}
\item FUN-02, FUN-45 (top of lines, velocity change) 
\item FUN-05, FUN-45 (bottom of lines, velocity change) 
\item FUN-06, FUN-19 (bouncing off walls, velocity change) 
\item FUN-23 (platform size affecting velocity) 
\end{itemize}
\item off-screen movement 

\begin{itemize}
\item FUN-03 (falling off (bottom of) screen) 
\item FUN-04 (moving up (top of) screen) 
\end{itemize}
\item collisions 

\begin{itemize}
\item FUN-11, FUN-22 (power-up collection, activation, -doesn\textquoteright{}t
affect movement-) 
\item FUN-03 (touching enemies falling off screen ending game) \Circle{}
FUN-46 (movement affecting score) 
\end{itemize}
\end{itemize}
\item Resources needed: 

\begin{itemize}
\item Relevant devices (iPhone/iPad, Android, facebook accessable PC and
web browser) 
\item Timing software 
\item Trajectory measuring software
\end{itemize}
\end{itemize}

\subsection{TES-1.1: Game Start State }
\begin{itemize}
\item Goal: Show that the avatar, at the start of a game, will start at
the top center of the user\textquoteright{}s screen and accelerate
in the -y direction only 

\begin{itemize}
\item Set a time bound (?) 
\end{itemize}
\item Testcases: 

\begin{itemize}
\item Try starting game with several different avatar/power up combinations 
\item Try pausing game, ending it, and then starting it again (does the
game state get cleared when starting new game?) 
\end{itemize}
\item Capture the output 

\begin{itemize}
\item Measure elapsed time 
\item Measure start location 
\item Measure acceleration 
\end{itemize}
\item Evaluate the output data 

\begin{itemize}
\item compare to expected data (in terms of the goals) 
\end{itemize}
\item Generate report 

\begin{itemize}
\item Produce summary of generated reports 
\item Evaluate summary with respect to goal
\end{itemize}
\end{itemize}

\subsection{TES-1.2: Pausing }
\begin{itemize}
\item Goal: Show that, while in a pause state, all moving objects halt in
place, but exiting the pause state will restore original movement 

\begin{itemize}
\item resumes gameplay in a coherent fashion that will not break the user\textquoteright{}s
immersion in the game 
\item objects originally moving in a certain trajectory will not suddenly
change trajectory for no reason (see TES-1.3) 
\item Game state will be saved (see TES 2.1) 
\end{itemize}
\item Testcases: 

\begin{itemize}
\item try tapping the pause/resume button on several different occasions
coinciding with various events 
\item try tapping the button at various time interval increments (ex. 50
ms, 1 s, 2 s, \ldots{} ) 
\end{itemize}
\item Capture the output 

\begin{itemize}
\item measure trajectories before entering/after leaving pause state 
\item measure elapsed response time and record qualitative response behavior 
\end{itemize}
\item Evaluate the numeric result 

\begin{itemize}
\item compare trajectories 
\item evaluate response time and behavior with those expected 
\end{itemize}
\item Generate report 

\begin{itemize}
\item Produce summary of generated reports with regard to data collection 
\item Evaluate summary with respect to goal
\end{itemize}
\end{itemize}

\subsection{TES-1.3: Bouncing Behavior }
\begin{itemize}
\item Goal: Show that the avatar will bounce off of objects in game, that
its trajectory changes in a fun and intuitive fashion 

\begin{itemize}
\item avatar will only change from -y to +y velocity when colliding with
the top of drawn lines (see FUN-02, FUN-45) 
\item no component of avatar\textquoteright{}s velocity will be affected
upon colliding with the bottoms of lines (see FUN-05, FUN-45) 
\item side walls (left and right) are \textquotedblleft{}solid\textquotedblright{}:
avatar will reverse x velocity upon collision (see FUN-06, FUN-19) 
\item shorter drawn platforms will give more of a boost than longer drawn
ones (see FUN-23) 
\end{itemize}
\item Testcases: 

\begin{itemize}
\item test various different environments in isolated conditions from the
bullets described in goals (ex. avatar with a velocity such that it
will collide with the top of a line only) 
\item test various different environments in various conditions from the
bullets described in goals (ex. avatar with a velocity such that it
will collide with the bottom of a line after colliding with the top
of a line) 
\item test the above two with different combinations of avatars, power-ups
that are in active/inactive statuses, arbitrarily placed lines (ranging
from extremely condensed to sparsely place ones) 
\end{itemize}
\item Capture the output 

\begin{itemize}
\item measure trajectories before/after each collision 
\item measure elapsed response time and record qualitative response behavior 
\end{itemize}
\item Evaluate the numeric result 

\begin{itemize}
\item organize data and compare trajectories with relevant goals 
\item evaluate response time and behavior with relevant goals 
\end{itemize}
\item Generate report 

\begin{itemize}
\item Produce summary of generated reports with regard to data collection 
\item Evaluate summary with respect to goal
\end{itemize}
\end{itemize}

\subsection{TES-1.4: Off-Screen Movement}

\subsection{TES-1.5: Collisions}

\subsection{TES-1.6: Movement Affecting Score}

\section{TSS-02 for Game State Mechanics }
\begin{itemize}
\item Goal: Determine that the mechanics that govern when the game is started
and lost work robustly and accurately.
\item Resources Required: 

\begin{itemize}
\item User Hours 
\item iPhone 
\item Android Phone 
\item Computer 
\end{itemize}
\item Structure of Test Plan: 

\begin{itemize}
\item Pause Button Pauses Game: FUN-18 

\begin{itemize}
\item Player can end, resume or restart game from Pause menu 
\end{itemize}
\item User loses when touching lethal hazard 

\begin{itemize}
\item Contact with enemies: ACC-07 
\item Falling through bottom: FUN-03 
\end{itemize}
\item Results of changing application state 

\begin{itemize}
\item Backgrounding the app mid-playthrough pauses game: FUN-40 
\item Exiting the app mid-playthrough loses the game: FUN-41 
\end{itemize}
\item In an optimal game the player climbs forever: ACC-05 

\begin{itemize}
\item Game should not transition to an end-state by climbing too high
\end{itemize}
\end{itemize}
\end{itemize}

\subsection{TES-2.1: Pause Menu }
\begin{itemize}
\item Goal: Show that the the Player can pause the game at any time during
a playthrough, which freezes game state. Show that the user can terminate
the game or restart their playthrough from the pause menu.
\item Test cases: 

\begin{itemize}
\item User starts game, presses pause button. 
\item From pause menu, user attempts to quit game. 
\item From pause menu, user attempts to restart game.
\end{itemize}
\item Capture the output 

\begin{itemize}
\item Collect data on how game state responded to operations 
\end{itemize}
\item Evaluate the result: 

\begin{itemize}
\item Pressing pause should bring up pause menu. 
\item Attempting to quit game should exit from game, to game over screen. 
\item Attempting to restart game should begin a new playthrough. 
\end{itemize}
\item Generate report 

\begin{itemize}
\item Produce summary of generated reports with regard to data collection 
\item Evaluate summary with respect to goal
\end{itemize}
\end{itemize}

\subsection{TES-2.2: Lethal Hazards }
\begin{itemize}
\item Goal: Show that contact of the player\textquoteright{}s avatar with
lethal hazards results in a lost game state as intended. 
\item Test cases: 

\begin{itemize}
\item User\textquoteright{}s avatar makes contact with enemy avatar. 
\item User\textquoteright{}s avatar makes contact with bottom of screen. 
\end{itemize}
\item Capture the output 

\begin{itemize}
\item Collect data on how game state responded to operations 
\end{itemize}
\item Evaluate the result: 

\begin{itemize}
\item Both cases should result in a lost game state. 
\end{itemize}
\item Generate report 

\begin{itemize}
\item Produce summary of generated reports with regard to data collection 
\item Evaluate summary with respect to goal
\end{itemize}
\end{itemize}

\subsection{TES-2.3: Change Application State }
\begin{itemize}
\item Goal: Show that changing the state of the application mid-playthrough
produces the intended result relative to the state of the current
playthrough. 
\item Test cases: 

\begin{itemize}
\item Background the Line Bounce application mid-playthrough. 
\item Exit the Line Bounce application mid-playthrough. 
\end{itemize}
\item Capture the output 

\begin{itemize}
\item Collect data on how game state responded to operations 
\end{itemize}
\item Evaluate the result: 

\begin{itemize}
\item Backgrounding the app mid-playthrough should pause the game, and bring
up the Pause Menu. \Circle{} Exiting the app mid-playthrough should
exit the game. 
\end{itemize}
\item Generate Report 

\begin{itemize}
\item Produce summary of generated reports with regard to data collection 
\item Evaluate summary with respect to goal
\end{itemize}
\end{itemize}

\subsection{TES-2.4: Climbing High }
\begin{itemize}
\item Goal: Show that climbing to a high altitude by itself does not change
the game state. 
\item Test cases: 

\begin{itemize}
\item In a playthrough, climb to 10{[}a{]},000 feet. 
\item In a playthrough, climb to 100,000 feet. 
\end{itemize}
\item Capture the output 

\begin{itemize}
\item Collect data on how game state responded to operations 
\end{itemize}
\item Evaluate the result: 

\begin{itemize}
\item In both cases, no matter how high the avatar climbs, game should not
end. 
\end{itemize}
\item Generate Report 

\begin{itemize}
\item Produce summary of generated reports with regard to data collection 
\item Evaluate summary with respect to goal
\end{itemize}
\end{itemize}

\section{TSS-03 for Menus{[}b{]} }
\begin{itemize}
\item Goal: Determine that the menus function according to the requirements.
(Note: The Store menus will be covered in the Store TSS.)
\item Resources Required: 

\begin{itemize}
\item User Hours 
\item iPhone 
\item Android Phone 
\item Computer 
\item Internet Connection 
\item Server
\end{itemize}
\item Structure of the Test Plan: 

\begin{itemize}
\item Menu features are present and work correctly 

\begin{itemize}
\item The Main Menu should feature many options: FUN-09 
\item The Settings Menu should allow users to change options: FUN-17 
\item The start screen will give the option of connecting to Facebook before
going to the Main Menu: FUN-33 
\item The user can logout of Facebook from the Settings Menu: FUN-36 
\item Pressing the pause button will bring up a menu that allows the user
to perform a few actions: FUN-39 
\item Buttons in menus should always perform desired functionality: ROB-08 
\end{itemize}
\item The menu will have fast response time: PER-04
\end{itemize}
\end{itemize}

\subsection{TES-3.1: Menu features are present and function correctly }
\begin{itemize}
\item Goal: Show that the menu features are present in the required locations
and perform the required functions across all platforms. 
\item Test cases: 

\begin{itemize}
\item Check for presence of all menu features 

\begin{itemize}
\item User navigates the game and menus 
\end{itemize}
\item Check functionality of all menu features 

\begin{itemize}
\item User tries to use menu features manually 
\item User writes and runs test code to check back-end functionality 
\end{itemize}
\end{itemize}
\item Capture the output 
\item Evaluate the result: 

\begin{itemize}
\item All required menu features should be present in the required locations 
\item All menu features should function as required 
\end{itemize}
\item Generate report
\end{itemize}

\subsection{TES-3.2: The menu will have fast response time }
\begin{itemize}
\item Goal: Show that the menu feature response times satisfy the performance
requirements stated in PER-04. 
\item Test cases: 

\begin{itemize}
\item Check the response time of each menu feature 

\begin{itemize}
\item User writes and runs test code that times each menu function 
\end{itemize}
\end{itemize}
\item Capture the output 
\item Evaluate the result: 

\begin{itemize}
\item All menu features should have a response time of under 0.25 seconds
(PER-04 as of 10/18/13) 
\end{itemize}
\item Generate report \end{itemize}
