% ====================================================
% Sample Chapter
% ====================================================

\chapter{Sample Chapter}

	\section{First things first}

One team member needs to place this template into your 
repository. All team members should then be able to check out a
working copy of this template. Before you start doing any work, read
through what is in this pdf, whilst at the same time looking at the
code. This is the best way to see how certain things are achieved
using \LaTeX.

%Also this code can have comments which do not appear in the generated
%document!

	\section{Structure of the files}

You will find that there are a number of files associated with this
document. Files containing the actual text have a `.tex'
extension. The file srs.tex is the ``main'' document --- that is, the
file that is fed into the \LaTeX compiler.



You will also find two directories. The `figs' directory should
contain all the images you are going to insert into the document. The
`stys' directory contains all the \LaTeX~style sheets. For this
submission you can ignore the style of the document and merely worry
about the content. In future documents that you will write and submit,
you can look at the styles if you wish to do so and adjust them to
suit your needs.

\section{Writing the content}

When you look at the source file for the text written here you can see the text that I have    written here  is all over the   place. \LaTeX~is not too fussed about the way you write the source. It will ignore things like redundant white spaces.










If you want to start another paragraph you simply leave a blank
like. However if you leave 34 blank lines, \LaTeX~still will do the
same thing it would have done had you left a single blank line.

Having said this, you should still try and organise the \LaTeX~source
files so that the content is readable. In other words, just like you
would do with C code, we expect \LaTeX~code to follow some very basic
`coding standards'.

To add a list in \LaTeX, you do not have to write down the
numbers. You simply tell \LaTeX~that you are now writing a list and
you identify each item in the list. Here is an example:


\begin{enumerate}
	\item first item
	\item second item
	\item some other item
\end{enumerate}

You can make text bold, by indicating which text it is, like
\textbf{here}. 

You can make text italicised, by indicating which text it is, like \textit{here}. 

You can also include tables in \LaTeX, however they can be a bit
tricky to format. An example table can be found in
Table~\ref{tab:example-table}. You can refer to \LaTeX~documentation
online. Look up section \ref{sec:more}. Here you can see that the
table number and section number are not hard-coded. We are using
\LaTeX's label-ref system, where you can label a section by placing
the \label{sec:title} tag next to it, and reference that section
number by calling it with \ref{sec:title}. This means that every time
the document is compiled, all the internal document references will
match.

%the ``t'' specifies that this table should be at the top of the page.
%You can specify other locations, such ``h'' for ``here'', ``b'' for
%``bottom of the page''
\begin{table}[t]
\centering
%This table has four columns, specifed by ``lcrp{5cm}''. This means
%that the first column will align to the left, the second to the
%centre, the third to the right, and the fourth will wrap the text,
%and be 5cm wide
% The ``|'' specifies that there should be a line between columns
\begin{tabular}{|l|c|r|p{5cm}|}
\hline
% the ampersand (&) divides columns
  \textbf{Left aligned} & \textbf{Centered} & \textbf{Right aligned} &
  \textbf{Wrapped text}\\
\hline
\hline
  1 & A & a & This text should wrap around to the next line\\
  2 & B & b & This line isn't long enough\\
\hline
\end{tabular}
\caption{An example table}
\label{tab:example-table}
\end{table}

\section{Finding out more} 
\label{sec:more}

You can find out more about \LaTeX~on the web. A nice simple guide can
be found at:

\url{http://www.ctan.org/tex-archive/info/lshort/english/lshort.pdf}

Other resources, such as \LaTeX\ IDEs and other manuals can be found
on the LMS, under {\em Resources} $\rightarrow$ {\em Latex}.

\section{Finally}

Once you start writing the document, you will need to commit your
contributions to the repository. It is important that you make sure
that the \LaTeX~document still compiles with the changes that you have
made. It is very annoying for others to have to track down what you
have broken.
