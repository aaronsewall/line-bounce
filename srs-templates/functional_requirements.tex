\chapter{Functional Requirements}
\section{Gameplay}
\subsection{FUN-01: Player starts the game from the top of the screen}
\textbf{Description}: The player should begin at the ceiling and accelerate
downward, leaving a few seconds to draw the first platform and begin
playing.

\subsection{FUN-02: The user draws lines to bounce their character upward}
\textbf{Description}: The player must draw line segments to bounce
off of. This is done by swiping your finger on the touch screen or
clicking and dragging with the mouse.

\subsection{FUN-03: The user loses when the avatar falls off the screen or runs
into a hazard}
\textbf{Description}: The player loses by falling off the bottom of
the screen, running into a hazard such as a hole, or being \textquoteleft{}killed\textquoteright{}
by an enemy. Once this happens they are taken to the Game Over menu.

\subsection{FUN-04: Screen moves upward in height during gameplay}
\textbf{Description}: The screen will only scroll up when the avatar
is moving vertically upward. The avatar will not pass the top of the
screen before the screen is readjusted.

\subsection{FUN-05: The avatar will bounce back down off of the bottom of platforms}
\textbf{Description}: If the user collides with a platform through
the bottom, it will pass through without any change in velocity.

\subsection{FUN-06: Side walls are solid}
\textbf{Description}: Unlike Doodle Jump where the avatar passes through
walls and exits the other side, in Line Bounce the avatar can bounce
off of the wall to avoid enemies and other hazards.

\subsection{FUN-07: The user has a limited length of line they can draw}
\textbf{Description}: When the player has drawn platforms of combined
length equal to the maximum and continues to draw, their previous
drawings are steadily erased. To visualize this, imagine the movements
of the snake in the classic game \textquoteleft{}Snake.\textquoteright{}

\subsection{FUN-08: Power-ups will be activated immediately upon acquisition }
\textbf{Description}: Power-ups will be activated as soon as soon
as the avatar acquires them.

\subsection{FUN-09: The user can pause during gameplay}
\textbf{Description}: At any point during gameplay the user can hit
the pause button allowing them bring up a separate screen menu to
suspend play. This menu does not allow the user to see the gameplay
in order to strategize. The pause menu allows the user to quit the
game, go to the store (loses progress), restart the current game,
and to be able to resume the game. There will be countdown when they
click resume until the game begins again.

\subsection{FUN-10: The avatar\textquoteright{}s horizontal velocity is flipped
when it collides with the side of the screen}
\textbf{Description}: The avatar\textquoteright{}s horizontal velocity
is flipped when the avatar collides with the boundaries of the screen,
while the vertical velocity remains the same. 

\subsection{FUN-11: Screen never moves downwards}
\textbf{Description}: The screen will only move upwards as the player
jumps and will never move downward to follow the player\textquoteright{}s
fall.

\subsection{FUN-12: The activated power-ups will be displayed on the screen}
\textbf{Description}: A list of the power-ups that are currently activated
(in the form of a row of icons), will be displayed at the top of the
screen during gameplay. These power-ups will be passive only and not
be clickable.

\subsection{FUN-13: Smaller platforms cause the player to bounce higher}
\textbf{Description}: The length of the platform will determine how
much upward acceleration the player gains from bouncing, with shorter
length equaling more bounce.

\subsection{FUN-14: A power-up will be collected when the avatar comes into 
contact with it}
\textbf{Description}: The avatar can pass completely through or just
graze the power-up to acquire it. Any form of contact between the
two sprites will count as a hit.

\subsection{FUN-15: Users receive experience points and coins after each play}
\textbf{Description}: At the end of the game, the user receives experience
points proportional to the height reached and number of enemies slain.

\subsection{FUN-16: The user has a level, determined by total experience points}
\textbf{Description}: Levels are determined by total experience points.
Gaining levels allows the user to purchase new items in the store.

\subsection{FUN-17: Enemy generation occurs randomly}
\textbf{Description}: An arbitrary number of enemies are generated
during gameplay every few seconds. The number of enemies is random,
but within a certain range that is actively changing during play. 

\subsection{FUN-18: Difficulty of game increases with height and level}
\textbf{Description}: The range described in FUN-30 is based on the
user\textquoteright{}s level as well as the current height at the
time of generation. As the player gets to higher levels (overall)
and heights (in a playthrough), both the maximum and the minimum of
the range will increase.

\subsection{FUN-19: The user\textquoteright{}s current score will be displayed
on the screen}
\textbf{Description}: During gameplay, the user\textquoteright{}s
current score will be displayed on the screen and updated in real
time to reflect the current playthrough. The user will initially have
a score of zero at the beginning of the game. This score will never
decrease. As the avatar rises and their height increases, the score
in the corner will change. 

\subsection{FUN-20: A drawn line will \textquotedblleft{}materialize\textquotedblright{}
upon completion}
\textbf{Description}: A line will only become solid, and allow the
avatar to bounce, when the user lifts his finger, and the second endpoint
of the line has been established.

\subsection{FUN-21: Lines drawn through an avatar will do nothing}
\textbf{Description}: Regardless of whether or not the user has completed
the line, the avatar will always pass through the line, if the user
has drawn the line through the avatar.

\section{Menu}

\subsection{FUN-22 Game Over menu will feature several courses of action}
\textbf{Description}: This menu will offer the player a chance to
start a new game, post a score to the Social page, go to the Store,
go to the leaderboards or return to the Main Menu.

\subsection{FUN-23 The Main Menu should feature many options}
\textbf{Description}: The menu should allow the user the choice to
begin a new game, check the leaderboards, access the in-game store,
invite friends through facebook, or adjust settings.

\subsection{FUN-24 The Online Store will feature many purchasable items}
\textbf{Description}: Users will be able to generate as well as buy
in-game currency and other items such as power-ups, new skins/stages,
new music, etc. This will represented by a list of categories that
users click on in order to purchase the item in that category. Each
item will have their price written next to them.

\subsection{FUN-25 The Settings menu should allow users to change options}
\textbf{Description}: The user should be able to log in and out of
Facebook, sync his/her data, change whether or not they get in-game
notifications, and reset their local history.

\subsection{FUN-26 The start screen will give the option of connecting to Facebook}
\textbf{Description}: There will be a button that links to the Facebook
login screen{[}c{]} so that the user can sign into the game. The user
can also press a \textquotedblleft{}Play Offline\textquotedblright{}
button to play the game without connecting to Facebook.

\subsection{FUN-27 The preplay screen which will allow the user to select items}
\textbf{Description}: On the preplay screen the user\textquoteright{}s
avatar will be displayed along with his/her stats. A list of the items
the user has already purchased (including power-ups, skins, etc.)
will be visible, and each item will have a corresponding check box
to allow the user to select it for their next playthrough.

\section{Settings}

\subsection{FUN-28 Facebook authentication will be required for purchases}
\textbf{Description}: The user can view the store\textquoteright{}s
contents without logging, in, however he/she will need to authenticate
via Facebook before being able to purchase any item. The user will
be able to buy coins through the Facebook store interface.

\subsection{FUN-29 Users can reset their in-game progress}
\textbf{Description}: Users can reset their progress, statistics and
achievements on both their facebook account and on their offline/local
account. This distinction will be made clear in the settings menu
so that users don\textquoteright{}t accidentally reset the wrong account.

\subsection{FUN-30 Resetting progress should be simple with user confirmation}
\textbf{Description}: The reset function should be simple and easy
and have user confirmation to prevent accidental reset.

\subsection{FUN-31 The user can logout of Facebook from the settings menu}
\textbf{Description}: The user can logout of Facebook and will be
able to continue to play in offline mode. A confirmation dialog will
pop up asking the user to verify that they wish to log out. The confirmation
dialog will also notify the user that their progress will be synced
to Facebook the next time they log in.

\subsection{FUN-32 Backgrounding the application mid-playthrough will pause
the game}
\textbf{Description}: Backgrounding the application (via a home button
press, for example) will pause the game. The state of the game will
be saved so that when the user returns he/she will be brought to the
pause menu and can resume play.

\subsection{FUN-33 Exiting the application mid-playthrough will end the playthrough}
\textbf{Description}: Exiting the application (e.g. closing the browser
window or quitting the application entirely) will completely lose
all of the current playthrough\textquoteright{}s progress

\section{Social}

\subsection{FUN-34 The user will be able to post a score to Facebook}
\textbf{Description}: The user will be able to post a score to Facebook
with a short message attached. This should be done without loading
the social page.

\subsection{FUN-35 The user must be able to access and post any previous scores}
\textbf{Description}: This can be done from a tab on the Social Features
page. The posts go to their Facebook wall.

\subsection{FUN-36 The user will receive coins for inviting friends to play}
\textbf{Description}: An sent invite will earn the user a few coins,
an accepted invitation will earn a much larger sum. 

\subsection{FUN-37 Users will be able to send power-ups to friends}
\textbf{Description}: Next to each name on the leaderboard, there
will be an arrow icon. Tapping it will pop-up a confirmation page
to send a power-up to that friend. After sending a power-up there
will be a cooldown period during which the user will be unable to
send power-ups to the same friend.

\subsection{FUN-38 The user will be alerted when they score higher than their
friends}
\textbf{Description}: After a playthrough, a popup screen (before
the game over screen) will notify the user if he/she scored higher
than a friend\textquoteright{}s personal best during that playthrough.

\subsection{FUN-39 The user will receive notifications about friends\textquoteright{}
activities}
\textbf{Description}: The user will only receive notifications about
activities involving the user directly. For example, the user will
receive a notification if a friend gifts them a power-up, or if the
friend surpasses their high score.

